\documentclass{article}

\usepackage{amsmath}
\usepackage{amssymb}

\usepackage[backend=biber,style=numeric]{biblatex}
\addbibresource{refs.bib}
\title{Simplified Climate Simulation}
\author{
    Felix Hellborg,
    Rasmus Pernesten
    \\
    \textbf{Handledare: Erik Swietlicki}
}
\date{
    \textbf{\date}
}
\begin{document}
  \maketitle


  \begin{abstract}
   Climate change is one of the biggest challenges of modern times, posing a threat comparable to that of nuclear war. Enormous resources are therefore allocated to preventing and mitigating the consequences of global warming. To tackle this issue effectively-and to avoid spending time and resources on ineffective measures-climate models are developed. With the aim of understanding how such a model can be constructed and what difficulties arise in the process, we have created a highly simplified climate simulation using programming and mathematical modelling to compute different aspects of a climate system.
   This model does not aim to predict exact measurements, but rather to reproduce real-world phenomena.
   Our goal with the model was to observe effects such as day–night cycles, seasonal variations, colder poles, and a warmer equator. Although the model failed to reproduce all of these features, it is thermally stable, and it clearly shows notable differences between day and night, summer and winter, as well as between equatorial and polar regions.
  \end{abstract}
  
  \section{Introduction}

  This paper aims to demonstrate how a simplified climate model can be constructed and what limitations such a model inevitably has. A simplified model cannot predict exact events such as weather patterns or detailed effects of climate change, but many of the underlying ideas are the same as those used in more complex models. Most components of this climate simulation could, in principle, be incorporated into a more accurate climate model if the complexity of the system were increased. For example, heat transfer in this simulation occurs solely through conduction, which is a strong simplification, since atmospheric convection plays a major role in real-world climate dynamics. More limitations of the model will be discussed at the end of the paper.

  Furthermore, the model is designed to test whether simplifying the atmosphere still produces a stable energy level and whether it can recreate real-world phenomena, or if such simplifications would instead lead to runaway effects or unrealistic behaviour such as extreme temperature differences. Since general trends are of greater interest than exact numerical values, several constants and parameters will be adjusted to yield realistic behaviour; more on this is explained in Section~2.1. This might appear questionable when discussing whether the mathematical models underlying the simulation function correctly or not, but we ask the reader to remember that the graphical output serves as an illustration of the model rather than an exact representation of physical reality. For example, the precise magnitude of the temperature increase due to greenhouse gases is less important than the qualitative observation that such an increase occurs; which our model successfully demonstrates.

  \subsubsection{Nomenclature}
   
  \section{Theory}
 
  \section{Solar System}
  The solar system model used in the simulation is a simplified model of our own solar system, consisting of only one planet. The parameters of the planet's orbit can be set arbitrarily, including period time $T$, time at perihelion $\tau_p$, eccentricity $e$. The planet's motion relative to the star (always located at the origin) adheres to Kepler's laws of planetary motion:
  $$x = a(\cos E - e)$$
  $$y = b\sin E = a \sqrt{1-e^2} \sin E$$
  where $a$ is the semi-major axis, $b$ is the semi-minor axis, and the eccentric anomaly $E$ is related to the mean anomaly $M$ by
  $$M = E - e\sin E$$
  where
  $$M \equiv n(t-\tau_p).$$
  
  Here, n denotes the average rate of sweep
  $$n \equiv \frac{2\pi}{T},$$
  and $t$ is the time in arbitrary units since start.

  The exact solution to the eccentric anomaly $E(t)$ lacks a general closed form, it is usually expressed as a Fourier series. For the purposes of this simulation, the exact solution is irrelevant, and there happens to be a very good numerical approximation for $E(t)$ developed by Seppo Mikkola in 1987. While the details of this method are beyond the scope of this paper, the method is presented in Appendix \ref{Mikkola}.
  
  \section{Heat Equation}

  The heat equation on a flat cartesian plane can be written as \cite{wikiheat}
  $$
    \partial_t T = \kappa\nabla^2 T = \kappa(\partial^2_{xx} T + \partial^2_{yy} T)
  $$
  where $\nabla^2$ denotes the Laplacian. In section \ref{sphere-coords}, we will generalize the heat equation further to be defined on a spherical surface using the Laplace-Beltrami operator. For now, however, we will use the simplest possible geometry to discuss how we can simulate heat diffusion in a discrete system.

  \subsection{Runge-Kutta Methods}
  Assuming we have discretesized the map into a heat matrix T at some time $t$, we can solve for T at time (t+h) using a system of RK4 calculations \cite{originalkutta} \cite{wikikutta}.

  Given an equation
  \begin{equation}
    \frac{dA}{dt} = F(A, t)
  \end{equation}
  where $A$ is a matrix and $F$ is a linear function in the basis of $A$, and an initial value of $A(t)$, we get

  \begin{equation}
    A(t+h) \approx \frac{h}{6}(k_1+k_2+k_3+k_4)
    \label{eq:rk4}
  \end{equation}
  where
  \begin{equation}
    k_1 = F(A(t)),
    \label{eq:k1}
  \end{equation}
  
  \begin{equation}
    k_2 = F(A(t) + \frac{h}{2}k_1),
    \label{eq:k2}
  \end{equation}
  
  \begin{equation}
    k_3 = F(A(t) + \frac{h}{2}k_2),
    \label{eq:k3}
  \end{equation}
  
  \begin{equation}
    k_4 = F(A(t) + hk_3).
    \label{eq:k4}
  \end{equation}
  The approximation given by RK4 perform significantly better than the familiar Euler method, especially at larger time-steps, allowing us to run the simulation at higher speeds.

  Note that for partial differential equations, we must also discretesize (approximate) derivatives as follows:

  \begin{equation}
    \partial^2_{x^ix^i} a_{ij} \approx \frac{-a_{i(j+1)} + 2a_{ij} - a_{i(j-1)}}{(\Delta x)^2}
  \end{equation}

  Here, $\partial^2_{x^kx^k} a_{ij}$ denotes the second derivative of element $a_{ij}$ with respect to coordinate $x^k$.

  

  \subsection{Spherical Coordinates}\label{sphere-coords}
  In the beginning of this chapter, we wrote the heat equation using the Laplacian $\nabla^2$, which works for geometries lacking any curvature where the metric tensor $g_{\mu\nu}$ is defined on the flat-space $\mathbb{R}^2$ Reimann manifold where the interval is
  $$ds^2 = dx^2 + dy^2.$$

  In general, we define the metric tensor on a manifold through the interval
  $$ds^2 = g_{\mu\nu}dx^\mu dx^\nu.$$
  (Note the usage of the Einstein summation convention.)

  The heat equation then becomes
  \begin{equation}
    \partial_t T = \kappa \Delta_g T
    \label{eq:General-Heat-Eq}
  \end{equation}
  where $\Delta_g$ denotes the Laplace-Beltrami operator which, when applied to some scalar function $f$, yields \cite{wikilaplace}
  \begin{equation}
    \Delta_g f = \frac{1}{\sqrt{|g|}} \partial_i(\sqrt{|g|}g^{ij}\partial_j f).
    \label{eq:Laplace-Beltrami}
  \end{equation}
  Here, $|g|$ is the absolute value of the determinant of the metric tensor. Again, note the usage of the Einstein summation convention.
  
  In our flat case:
  $$ g_{\mu\nu} =
  \begin{pmatrix}
    1 & 0\\
    0 & 1
  \end{pmatrix}
  $$
  and $|g|$ is $1$. Thus, we can calculate $\Delta_g f$ from equation \ref{eq:Laplace-Beltrami} as
  
  $$\Delta_g f = \partial^2_{x^1x^1} f + \partial^2_{x^2x^2} f$$
  or in cartesian coordinates
  
  $$\Delta_g f = \partial^2_{xx} f + \partial^2_{yy} f.$$

  We're interested in the $S^2$ Riemann manifold where 

  $$g = 
  \begin{pmatrix}
    R^2 & 0 \\
    0 & R^2\sin^2{\theta}
  \end{pmatrix}.
  $$
  Because
  $$g_{ij} = 0 \quad \forall \quad i \neq j$$
  we find
  $$|g| = R^4\sin^2{\theta} \Rightarrow \sqrt{|g|} = R^2\sin{\theta} \quad \forall \quad \theta \in [0, \pi]$$
  and
  $$g^{ii} = \frac{1}{g_{ii}}.$$
  
  From equation \ref{eq:Laplace-Beltrami}, we find 

  $$\Delta_{S^2}f = \frac{1}{R^2\sin{\theta}}(\partial_\theta(R^2\sin{\theta}(g^{\theta\theta}\partial_\theta f)) + (\partial_\phi R^2\sin{\theta}(g^{\phi\phi}\partial_\phi f))).$$
  Here, we've used the fact that $g^{\theta\phi}=g^{\phi\theta}=0$ to simplify the expression. We now substitute our values for $g^{\theta\theta}$ and $g^{\phi\phi}$ to find
  $$\Delta_{S^2}f = \frac{1}{R^2}(\cot{\theta}\partial_\theta f + \partial^2_{\theta\theta} f + \frac{1}{\sin^2{\theta}}\partial^2_{\phi\phi} f).$$
  
  Now, we can rewrite equation \ref{eq:General-Heat-Eq} as
  $$\partial_t T = \kappa \frac{1}{R^2}(\cot{\theta}\partial_\theta T + \partial^2_{\theta\theta} T + \frac{1}{\sin^2{\theta}}\partial^2_{\phi\phi} T)$$
  which we discretesize using
  $$\partial_{x^k} T_{ij} \approx \frac{T_{ij}-T_{i(j-1)}}{\Delta x^k}$$
  $$\partial^2_{x^kx^k} T_{ij} \approx \frac{-T_{(i+1)j} + 2T_{ij} - T_{(i-1)j}}{(\Delta x^k)^2}$$
  for coordinate $x^k$.


  \subsection{Atmosphere}
  The atmosphere is arguably the most important and most complex part of any climate model. Without it, calculations and simulations become very simple, and everything can be simplified to black bodies or gray bodies. However, when the object in question is not a stellar body like the Moon, the atmosphere must be taken into account. There are several methods for doing this, each with its own strengths and weaknesses.  
  \subsubsection{Monte Carlo Method}
  The Monte Carlo method is the most accurate approach for simulating light traveling through a medium such as the atmosphere. In this method, the path of each photon is simulated individually, and every interaction with the atmosphere is calculated separately. It is therefore highly flexible and can be applied to any medium and at any wavelength, but it is unrealistic for large-scale simulations due to its heavy computational cost. [NASA 1986] 
  \subsubsection{Flux Transmission Towards the Earth}
  Since the Monte Carlo method is far too computationally heavy to run within a reasonable time on a planet-sized system, simplifications must be made. We begin by separating the incoming solar radiation from the outgoing thermal radiation emitted by the planet. In this chapter, we discuss how we model the light approaching the Earth and what happens to it as it passes through the atmosphere.
  \\
  One important simplification is that we are only interested in the top of the atmosphere, where $\tau_{\nu} = 0$, and the bottom of the atmosphere, where $\tau_{\nu} = \tau_{\nu}^*$. Everything that happens between these two heights is separated into three cases.
   
  \begin{enumerate}
    \item The direct flux component is modeled as
      \begin{equation}
      \pi \mu_0 F_0 e^{-\tau^{*}/\mu_0}
       \label{eq:direct_compnente}
      \end{equation}

    \item The diffuse transmission component, which is scattered multiple times before reaching the surface, is modeled as
      \begin{equation}
      \pi \mu_0 F_0 t(-\mu_0)
        \label{eq:difused_componente}
      \end{equation}

    \item The diffuse component that is scattered multiple times and reflected between the atmosphere and the surface, which will be adressed soon.
  \end{enumerate}
  Together, equations~\ref{eq:direct_compnente} and~\ref{eq:difused_componente} can be combined into

\begin{equation}
    T_r(-\mu_0) = t(-\mu) + e^{-\tau^{*}/\mu_0}.
\end{equation}

Here, $t(-\mu)$ is approximated by $\frac{\omega \tau^{*}}{2\mu_0}$, where $\omega$ is given by
\[
\omega = \frac{\text{diffusion}}{\text{diffusion} + \text{absorption}}.
\]
If this expression is multiplied by the incoming flux, we obtain

\begin{equation}
    \pi \mu_0 F_0 T_r(-\mu_0)
    \label{eq:direct-flux-tr}
\end{equation}

since the solar constant is $F_0 \pi$ and $\mu_0 = \cos(\theta_0)$, where $\theta_0$ is the angle of the incoming radiation measured from the local surface normal. The factor $\pi \mu_0 F_0$ therefore represents the flux at a given point as a function of the solar zenith angle.

This expression applies before any reflections from the surface are taken into account. Because there are theoretically an infinite number of such reflections, they are treated as the convergent series

\begin{equation}
    \pi \mu_0 F_0 T_r(-\mu_0)
    \left[
        \rho_s \bar{r} 
        + \rho_s^2 \bar{r}^2 
        + \rho_s^3 \bar{r}^3 
        + \dots
    \right]
    \label{eq:multiple-scattering-series}
\end{equation}

where $\bar{r}$ is the atmospheric albedo without the surface contribution, and $\rho_s$ is the surface albedo. 
Note that this infinite-reflection model is valid only when the surface is Lambertian, meaning that reflections are assumed to be equally likely in all directions. Each term in the series represents one additional surface reflection followed by atmospheric scattering: the first term corresponds to a single reflection, the second term to two reflections, and so on. 
Fortunately, this series is convergent, making it straightforward to evaluate numerically.

  \begin{equation}
  \begin{aligned}
    F_T(\mu_0)
        &= \pi \mu_0 F_0 T_r(-\mu_0)
           \left[ 1 + \rho_s \bar{r} + \rho_s^2 \bar{r}^2 + \rho_s^3 \bar{r}^3 + \dots \right] \\
        &= \frac{\pi \mu_0 F_0 T_r(-\mu_0)}{1 - \rho_s \bar{r}}
  \end{aligned}
  \label{eq:ft-series}
  \end{equation}

\subsection{Simulation}
 The simulation built during this project is written in Rust using the Bevy game engine.
  
  \subsection{Units}
  For the purposes of this project, the units of measurements are not the regular SI-units. Because we're interested in the trends, and the relationships between parameters and outcomes, the values of the simulation parameters have been set such that they yield interesting and realistic results with respect to internal consistency (within the units of measurements we're using, whatever they may be).

  When in the course of reading this report, the reader comes across the usage of distance units [D.U.] or temperature units [T.U.], they should be imagined as corresponding to meters [m] and Kelvin [K] respectively with some arbitrary scaling factor. Note that temperature will always be given in an absolute scale (i.e. starting at zero).

  The visual representation should not be thought of as having a realistic scale. The size of a planet with respect to the distance to its host star is so small as to be impossible to see in a realistic relative scale. The same thing is true for the relative difference in size between the planet and the host star itself (the diameter ratio between the earth and the sun is roughly 110:1 \cite{wikisun}).
  





  \newpage
  \appendix
  \section{Mikkola's Approximation}\label{Mikkola}
  Here follows the full calculation by which the eccentric anomaly $E(t)$ can be approximated using Mikkola's method. This paper only presents the elliptical case and does not provide any proof or explanation to any of these steps. The curious reader is referred to the original paper \cite{mikkola}.

  \subsection{Formula}
  Assuming the value of the mean anomaly $M(t)$ is known, Kepler's equation can (in the elliptical case) be rewritten as
  
  $$E(t) = M(t) + e(3s-4s^3)$$
  where
  $$s \equiv \sin{E(t)/3}.$$

  The trick lies in approximating $s$, which can be done by setting
  $$\alpha \equiv (1-e)/(4e+\frac{1}{2})$$
  and
  $$\beta \equiv \frac{1}{2}M/(4e+\frac{1}{2}).$$

  Now let
  $$z = (\beta \pm \sqrt{\beta^2+\alpha^3})^{1/3},$$
  choosing the sign of the root term to match the sign of $\beta$.
  
  We can, by means of a series expansion approximate $s$ by
  $$s \approx z - \frac{\alpha}{z}.$$

  Note that $M$ should be chosen as to be in the range $(-\pi, \pi]$ such that it is as close as possible to $0$, thus yielding the least error from the series expansion.


 

  \newpage
  \printbibliography
\end{document}
